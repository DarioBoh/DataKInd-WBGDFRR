
\documentclass[12pt, a4paper, oneside, headinclude, footinclude]{article}

%----------------------------------------------------------------------------------------
%	REQUIRED PACKAGES
%----------------------------------------------------------------------------------------

\usepackage[
nochapters, 
beramono, 
eulermath,
pdfspacing, 
dottedtoc 
]{classicthesis} 

\usepackage{arsclassica} 
\usepackage{listings} 
\usepackage{minted} 
\usepackage[T1]{fontenc} 
\usepackage[utf8]{inputenc} 
\usepackage{graphicx} 
\graphicspath{{Figures/}} 
\usepackage{enumitem} 
\usepackage{lipsum} 
\usepackage{subfig} 
\usepackage{amsmath,amssymb,amsthm} 
\usepackage{varioref} 

%----------------------------------------------------------------------------------------
%	THEOREM STYLES
%---------------------------------------------------------------------------------------

\theoremstyle{definition} 
\newtheorem{definition}{Definition}

\theoremstyle{plain} 
\newtheorem{theorem}{Theorem}

\theoremstyle{remark} 

%----------------------------------------------------------------------------------------
%	HYPERLINKS
%---------------------------------------------------------------------------------------

\hypersetup{
colorlinks=true, breaklinks=true, bookmarks=true,bookmarksnumbered,
urlcolor=webbrown, linkcolor=RoyalBlue, citecolor=webgreen, 
pdftitle={}, 
pdfauthor={\textcopyright}, 
pdfsubject={}, 
pdfkeywords={}, 
pdfcreator={pdfLaTeX}, 
pdfproducer={LaTeX with hyperref and ClassicThesis} 
}


\title{\normalfont\spacedallcaps{Image analysis for disaster recovery, A DataKind report for the World Bank GFDRR}}

\author{\spacedlowsmallcaps{Krishna Bhogaonker \& Patrick Doupe}} 

\date{} 

\begin{document}

\renewcommand{\sectionmark}[1]{\markright{\spacedlowsmallcaps{#1}}} 
\lehead{\mbox{\llap{\small\thepage\kern1em\color{halfgray} \vline}\color{halfgray}\hspace{0.5em}\rightmark\hfil}} 

\pagestyle{scrheadings} 

%----------------------------------------------------------------------------------------
%	TABLE OF CONTENTS & LISTS OF FIGURES AND TABLES
%----------------------------------------------------------------------------------------

\maketitle 

\setcounter{tocdepth}{2}

\tableofcontents 

\listoffigures 

\listoftables 

\listoflistings

%----------------------------------------------------------------------------------------
%	ABSTRACT
%----------------------------------------------------------------------------------------

\section*{Abstract} 


We discuss how the World Bank can use machine learning and satellite images to
improve disaster relief efforts. We include a review of image analysis with
convolutional neural networks. These networks are illustrated with code
examples using the Keras deep learning library. 


%----------------------------------------------------------------------------------------
%	AUTHOR AFFILIATIONS
%----------------------------------------------------------------------------------------

%\let\thefootnote\relax\footnotetext{* \textit{}}

%\let\thefootnote\relax\footnotetext{\textsuperscript{1} \textit{}}

%----------------------------------------------------------------------------------------

\newpage 

%----------------------------------------------------------------------------------------
%	INTRODUCTION
%----------------------------------------------------------------------------------------

\section{Introduction}

The review of deep learning~\cite{lecun2015deep}.

\subsection{Terms of reference or questions to be answered}

\subsection{Main recommendations}


%----------------------------------------------------------------------------------------
%	METHODS
%----------------------------------------------------------------------------------------

\section{Frameworks}

If we present methods, it would be good to introduce frameworks up front.

\subsection{Keras}


\begin{minted}{python}
from keras.layers import Input
from keras.layers import Dense
from keras.models import Model

# This returns a tensor
inputs = Input(shape=(10,))
predictions = Dense(1, activation='softmax')(inputs)
model = Model(inputs=inputs, outputs=predictions)
model.compile(optimizer='rmsprop',
              loss='categorical_crossentropy',
              metrics=['accuracy'])
# starts training
model.fit(data, labels)  
\end{minted}



\subsection{Tensorflow, PyTorch, Caffe, MXNet, etc.}

Links to sites and  blog articles

\section{Methods}

\subsection{Why convolutional neural networks}

1. Pixels are not independent of one another
2. With pictures, only relative position matters [Spatially stationary
statistics]
3. Much fewer neurons.

Results of image analysis

\subsubsection{If you understand VGG-net, you're 80 per cent there}

VGG net is super simple and easy to explain.

\section{Literature}

\subsection{Image classification}

LeNet

Krizhevsky ImageNet~\cite{NIPS2012_4824}
\begin{itemize}
    \item Used dropout to prevent overfitting
    \item We start to go deep: 5 convolutional layers
    \item Got best results to that date on a hard problem
\end{itemize}


VGG Net~\cite{SimonyanZ14a}
\begin{itemize}
    \item Increased depth (16--19 layers) with smaller filters
    \item First/Second places at ImageNet2014
    \item Widely used as a base :: need references.
\end{itemize}

ResNet~\cite{he2016deep}

\begin{itemize}
    \item Residual learning framework
    \item Degradation became an issue with deep models. This makes no sense,
        you could just have a shallower model with identity mappings. So
        residual networks contain the identity mapping.
    \item Crazy large :: 152 layers but with lower 'complexity'
    \item Then state of the art.
\end{itemize}

\subsection{Object detection}
Using regression to do this~\cite{NIPS2013_5207}

RCNN~\cite{Girshick2014, Girshick2015}

Faster RCNN~\cite{Ren2017}

Mask RCNN~\cite{he2017}

R-FCN~\cite{NIPS2016_6465}

YOLO~\cite{redmon2016yolo}

SSD~\cite{liu2016ssd}

    
\subsection{Image segmentation}

First fully convnet segmenter~\cite{long2015fully}

Some more~\cite{chen2018, NIPS2015_5852}


\subsection{Analysis with satellite images}

Mexico~\cite{babenko2017poverty} (There are two WB co-authors)

Poverty mapping~\cite{Jean790}

Population~\cite{doupe2016, robinson2017}

Private sector~\cite{facebook, cnn_orbital}

\section{Satellite data}

\subsection{Free sources}

LANDSAT, Sentinel, 

Quasi free: Bing

\subsection{Expensive sources}

Digital Globe, Planet Labs, Own drone data

\subsection{Labelled data}

\section{Examples}

\subsection{Building detection}

\section{Recommendations}

%----------------------------------------------------------------------------------------
%	BIBLIOGRAPHY
%----------------------------------------------------------------------------------------

\renewcommand{\refname}{\spacedlowsmallcaps{References}} 

\bibliographystyle{unsrt}

\bibliography{review.bib}

%----------------------------------------------------------------------------------------

\end{document}
